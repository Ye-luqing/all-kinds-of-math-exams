\documentclass[a4paper]{article}
\usepackage{amsmath,amsfonts,amsthm,amssymb}
\usepackage{bm}
\usepackage{draftwatermark,euler}
\SetWatermarkText{http://blog.sciencenet.cn/u/Yaleking}%设置水印文字
\SetWatermarkLightness{0.8}%设置水印亮度
\SetWatermarkScale{0.35}%设置水印大小
\usepackage{hyperref}
\usepackage{geometry}
\usepackage{yhmath}
\usepackage{pstricks-add}
\usepackage{framed,mdframed}
\usepackage{graphicx,color} 
\usepackage{mathrsfs,xcolor} 
\usepackage[all]{xy}
\usepackage{fancybox} 
\usepackage{xeCJK}
\newtheorem*{theo}{定理}
\newtheorem*{exe}{题目}
\newtheorem*{rem}{评论}
\newmdtheoremenv{lemma}{引理}
\newmdtheoremenv{corollary}{推论}
\newmdtheoremenv{example}{例}
\newenvironment{theorem}
{\bigskip\begin{mdframed}\begin{theo}}
    {\end{theo}\end{mdframed}\bigskip}
\newenvironment{exercise}
{\bigskip\begin{mdframed}\begin{exe}}
    {\end{exe}\end{mdframed}\bigskip}
\geometry{left=2.5cm,right=2.5cm,top=2.5cm,bottom=2.5cm}
\setCJKmainfont[BoldFont=SimHei]{SimSun}
\renewcommand{\today}{\number\year 年 \number\month 月 \number\day 日}
\newcommand{\D}{\displaystyle}\newcommand{\ri}{\Rightarrow}
\newcommand{\ds}{\displaystyle} \renewcommand{\ni}{\noindent}
\newcommand{\ov}{\overrightarrow}
\newcommand{\pa}{\partial} \newcommand{\Om}{\Omega}
\newcommand{\om}{\omega} \newcommand{\sik}{\sum_{i=1}^k}
\newcommand{\vov}{\Vert\omega\Vert} \newcommand{\Umy}{U_{\mu_i,y^i}}
\newcommand{\lamns}{\lambda_n^{^{\scriptstyle\sigma}}}
\newcommand{\chiomn}{\chi_{_{\Omega_n}}}
\newcommand{\ullim}{\underline{\lim}} \newcommand{\bsy}{\boldsymbol}
\newcommand{\mvb}{\mathversion{bold}} \newcommand{\la}{\lambda}
\newcommand{\La}{\Lambda} \newcommand{\va}{\varepsilon}
\newcommand{\be}{\beta} \newcommand{\al}{\alpha}
\newcommand{\dis}{\displaystyle} \newcommand{\R}{{\mathbb R}}
\newcommand{\N}{{\mathbb N}} \newcommand{\cF}{{\mathcal F}}
\newcommand{\gB}{{\mathfrak B}} \newcommand{\eps}{\epsilon}
\renewcommand\refname{参考文献}\renewcommand\figurename{图}
\usepackage[]{caption2} 
\renewcommand{\captionlabeldelim}{}
\setlength\parindent{0pt}
\begin{document}
\title{\huge{\bf{杭州师范大学一道解析几何期末试题}}} \author{\small{叶卢庆\footnote{叶卢庆(1992---),男,杭州师范大学理学院数学与应用数学专业本科在读,E-mail:yeluqingmathematics@gmail.com}}}
\maketitle
\begin{exercise}
  证明:四面体每条棱与对棱上的中点所决定的$6$个平面交于一点.
\end{exercise}
\begin{proof}[\textbf{证明}]
如图,以$\{\overrightarrow{AB},\ov{AC},\ov{AD}\}$为仿射标架,可得经
过棱$AD$和$BC$中点的平面方程为
\begin{equation}
  \label{eq:1}
  x=y.
\end{equation}
经过棱$BC$和$AD$中点的平面方程为
\begin{equation}
  \label{eq:4}
  x+y+2z=1.
\end{equation}
经过棱$AC$和$BD$中点的平面方程为
\begin{equation}
  \label{eq:2}
  x=z.
\end{equation}
经过棱$BD$和$AC$中点的平面方程为
\begin{equation}
  \label{eq:5}
  x+z+2y=1.
\end{equation}
经过棱$AB$和$DC$中点的平面方程为
\begin{equation}
  \label{eq:3}
  y=z.
\end{equation}
经过棱$DC$和$AB$中点的平面方程为
\begin{equation}
  \label{eq:6}
  y+z+2x=1.
\end{equation}
下面我们证明上面$6$个方程联立只有一个解即可.由于$x=y=z$,因此
$x=y=z=\frac{1}{4}$.可见$6$个平面交于点
$(\frac{1}{4},\frac{1}{4},\frac{1}{4})$.换用向量的语言,$6$个平面交于唯
一一点,该点可以用向量表达成
$$
\frac{1}{4}\overrightarrow{AB}+\frac{1}{4}\ov{AC}+\frac{1}{4}\ov{AD}.
$$
\begin{figure}[h]\label{fig:1}
  \centering
\includegraphics[width=0.6\textwidth]{/home/luqing/Desktop/files/杭州师范大学一道解析几何期末试题.png}.
  \caption{}
\end{figure}
\end{proof}
\end{document}
