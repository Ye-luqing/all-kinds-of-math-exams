\documentclass[a4paper]{article}
\usepackage{amsmath,amsfonts,amsthm,amssymb}
\usepackage{bm}
\usepackage{draftwatermark,euler}
\SetWatermarkText{http://blog.sciencenet.cn/u/Yaleking}%设置水印文字
\SetWatermarkLightness{0.8}%设置水印亮度
\SetWatermarkScale{0.35}%设置水印大小
\usepackage{hyperref}
\usepackage{geometry}
\usepackage{yhmath}
\usepackage{pstricks-add}
\usepackage{framed,mdframed}
\usepackage{graphicx,color} 
\usepackage{mathrsfs,xcolor} 
\usepackage[all]{xy}
\usepackage{fancybox} 
\usepackage{xeCJK}
\newtheorem*{theo}{定理}
\newtheorem*{exe}{题目}
\newtheorem*{rem}{评论}
\newmdtheoremenv{lemma}{引理}
\newmdtheoremenv{corollary}{推论}
\newmdtheoremenv{example}{例}
\newenvironment{theorem}
{\bigskip\begin{mdframed}\begin{theo}}
    {\end{theo}\end{mdframed}\bigskip}
\newenvironment{exercise}
{\bigskip\begin{mdframed}\begin{exe}}
    {\end{exe}\end{mdframed}\bigskip}
\geometry{left=2.5cm,right=2.5cm,top=2.5cm,bottom=2.5cm}
\setCJKmainfont[BoldFont=SimHei]{SimSun}
\renewcommand{\today}{\number\year 年 \number\month 月 \number\day 日}
\newcommand{\D}{\displaystyle}\newcommand{\ri}{\Rightarrow}
\newcommand{\ds}{\displaystyle} \renewcommand{\ni}{\noindent}
\newcommand{\ov}{\overrightarrow}
\newcommand{\pa}{\partial} \newcommand{\Om}{\Omega}
\newcommand{\om}{\omega} \newcommand{\sik}{\sum_{i=1}^k}
\newcommand{\vov}{\Vert\omega\Vert} \newcommand{\Umy}{U_{\mu_i,y^i}}
\newcommand{\lamns}{\lambda_n^{^{\scriptstyle\sigma}}}
\newcommand{\chiomn}{\chi_{_{\Omega_n}}}
\newcommand{\ullim}{\underline{\lim}} \newcommand{\bsy}{\boldsymbol}
\newcommand{\mvb}{\mathversion{bold}} \newcommand{\la}{\lambda}
\newcommand{\La}{\Lambda} \newcommand{\va}{\varepsilon}
\newcommand{\be}{\beta} \newcommand{\al}{\alpha}
\newcommand{\dis}{\displaystyle} \newcommand{\R}{{\mathbb R}}
\newcommand{\N}{{\mathbb N}} \newcommand{\cF}{{\mathcal F}}
\newcommand{\gB}{{\mathfrak B}} \newcommand{\eps}{\epsilon}
\renewcommand\refname{参考文献}\renewcommand\figurename{图}
\usepackage[]{caption2} 
\renewcommand{\captionlabeldelim}{}
\setlength\parindent{0pt}
\begin{document}
\title{\huge{\bf{杭州师范大学解析几何一道期末试题}}} \author{\small{叶卢庆\footnote{叶卢庆(1992---),男,杭州师范大学理学院数学与应用数学专业本科在读,E-mail:yeluqingmathematics@gmail.com}}}
\maketitle
\begin{exercise}
  证明:如果三个矢量$\mathbf{a}\times \mathbf{b}$,$\mathbf{b}\times
  \mathbf{c}$,$\mathbf{c}\times \mathbf{a}$共面,则它们也共线.
\end{exercise}
\begin{proof}[\textbf{证明}]
根据轮换对称性,不妨设
$$
\mathbf{a}\times \mathbf{b}=\mu \mathbf{b}\times \mathbf{c}+\lambda
\mathbf{c}\times \mathbf{a}.
$$
则
$$
\mathbf{a}\times \mathbf{b}=\mu \mathbf{b}\times \mathbf{c}-\lambda
\mathbf{a}\times \mathbf{c}=(\mu b-\lambda \mathbf{a})\times \mathbf{c}.
$$
于是,
$$
(\mathbf{a}\times \mathbf{b})\cdot \mathbf{c}=((\mu b-\lambda
\mathbf{a})\times \mathbf{c})\cdot \mathbf{c}=0,
$$
这说明向量$\mathbf{a},\mathbf{b},\mathbf{c}$共面.设
$$
\mathbf{c}=u \mathbf{a}+v\mathbf{b},
$$
则$\mathbf{b}\times \mathbf{c}=u \mathbf{b}\times
\mathbf{a},\mathbf{c}\times \mathbf{a}=v \mathbf{b}\times
\mathbf{a}$.可见,它们的确共线.
\end{proof}
\end{document}
