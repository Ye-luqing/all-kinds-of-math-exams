\documentclass[a4paper]{article}
\usepackage{amsmath,amsfonts,amsthm,amssymb}
\usepackage{bm}
\usepackage{euler}
\usepackage{hyperref}
\usepackage{geometry}
\usepackage{yhmath}
\usepackage{pstricks-add}
\usepackage{framed,mdframed}
\usepackage{graphicx,color} 
\usepackage{mathrsfs,xcolor} 
\usepackage[all]{xy}
\usepackage{fancybox} 
\usepackage{xeCJK}
\newtheorem*{theo}{定理}
\newtheorem*{exe}{题目}
\newtheorem*{rem}{评论}
\newmdtheoremenv{lemma}{引理}
\newmdtheoremenv{corollary}{推论}
\newmdtheoremenv{example}{例}
\newenvironment{theorem}
{\bigskip\begin{mdframed}\begin{theo}}
    {\end{theo}\end{mdframed}\bigskip}
\newenvironment{exercise}
{\bigskip\begin{mdframed}\begin{exe}}
    {\end{exe}\end{mdframed}\bigskip}
\geometry{left=2.5cm,right=2.5cm,top=2.5cm,bottom=2.5cm}
\setCJKmainfont[BoldFont=SimHei]{SimSun}
\renewcommand{\today}{\number\year 年 \number\month 月 \number\day 日}
\newcommand{\D}{\displaystyle}\newcommand{\ri}{\Rightarrow}
\newcommand{\ds}{\displaystyle} \renewcommand{\ni}{\noindent}
\newcommand{\ov}{\overrightarrow}
\newcommand{\pa}{\partial} \newcommand{\Om}{\Omega}
\newcommand{\om}{\omega} \newcommand{\sik}{\sum_{i=1}^k}
\newcommand{\vov}{\Vert\omega\Vert} \newcommand{\Umy}{U_{\mu_i,y^i}}
\newcommand{\lamns}{\lambda_n^{^{\scriptstyle\sigma}}}
\newcommand{\chiomn}{\chi_{_{\Omega_n}}}
\newcommand{\ullim}{\underline{\lim}} \newcommand{\bsy}{\boldsymbol}
\newcommand{\mvb}{\mathversion{bold}} \newcommand{\la}{\lambda}
\newcommand{\La}{\Lambda} \newcommand{\va}{\varepsilon}
\newcommand{\be}{\beta} \newcommand{\al}{\alpha}
\newcommand{\dis}{\displaystyle} \newcommand{\R}{{\mathbb R}}
\newcommand{\N}{{\mathbb N}} \newcommand{\cF}{{\mathcal F}}
\newcommand{\gB}{{\mathfrak B}} \newcommand{\eps}{\epsilon}
\renewcommand\refname{参考文献}\renewcommand\figurename{图}
\usepackage[]{caption2} 
\renewcommand{\captionlabeldelim}{}
\setlength\parindent{0pt}
\begin{document}
\title{\huge{\bf{杭州师范大学一道解析几何期末试题}}} \author{\small{叶卢庆\footnote{叶卢庆(1992---),男,杭州师范大学理学院数学与应用数学专业本科在读,E-mail:yeluqingmathematics@gmail.com}}}
\maketitle
\begin{exercise}
  已知$\triangle ABC$,点$O$是$\triangle ABC$的外心,用
  $\ov{AB},\ov{BC},\ov{CA}$表示向量$\ov{AO}$.
\end{exercise}
\begin{proof}[\textbf{解}]
设
$$
\ov{AO}=\lambda_1 \ov{AB}+\lambda_2\ov{AC},
$$
则
$$
\ov{BO}=\ov{BA}+\ov{AO}=-\ov{AB}+\ov{AO}=(\lambda_1-1)\ov{AB}+\lambda_2\ov{AC}.
$$
$$
\ov{CO}=\ov{CA}+\ov{AO}=\lambda_1\ov{AB}+(\lambda_2-1)\ov{AC}.
$$
由于是外心,因此
$$
\ov{AO}\cdot \ov{AO}=\ov{BO}\cdot \ov{BO},
$$
即
\begin{equation}
  \label{eq:1}
  (\lambda_1 \ov{AB}+\lambda_2\ov{AC})\cdot (\lambda_1
  \ov{AB}+\lambda_2\ov{AC})=((\lambda_1-1)\ov{AB}+\lambda_2\ov{AC})\cdot ((\lambda_1-1)\ov{AB}+\lambda_2\ov{AC}).
\end{equation}
也即
\begin{equation}
  \label{eq:2}
  \lambda_1\ov{AB}^2+\lambda_2\ov{AB}\cdot
  \ov{AC}=\frac{1}{2}\ov{AB}^2.
\end{equation}
同理,可得
\begin{equation}
  \label{eq:3}
  \lambda_1\ov{AB}\cdot \ov{AC}+\lambda_2\ov{AC}^2=\frac{1}{2}\ov{AC}^2.
\end{equation}
解得
$$
\lambda_2=\frac{\ov{AC}^2\ov{AB}^2-\ov{AB}^2(\ov{AB}\cdot
  \ov{AC})}{2[\ov{AB}^2\ov{AC}^2-(\ov{AB}\cdot
  \ov{AC})^2]}=\frac{\ov{AB}^2(\ov{AC}\cdot \ov{BC})}{2[\ov{AB}^2\ov{AC}^2-(\ov{AB}\cdot
  \ov{AC})^2]},
$$
$$
\lambda_1=\frac{\ov{AC}^2\ov{AB}^2-\ov{AC}^2(\ov{AB}\cdot
  \ov{AC})}{2[\ov{AB}^2\ov{AC}^2-(\ov{AB}\cdot \ov{AC})^2]}=\frac{-\ov{AC}^2(\ov{AB}\cdot\ov{BC})}{2[\ov{AB}^2\ov{AC}^2-(\ov{AB}\cdot
  \ov{AC})^2]}.
$$
可见,
\begin{align*}
\ov{AO}&=\frac{-\ov{CA}^2(\ov{AB}\cdot\ov{BC})}{2[\ov{AB}^2\ov{CA}^2-(\ov{AB}\cdot
  \ov{CA})^2]}\ov{AB}-\frac{\ov{AB}^2(\ov{CA}\cdot \ov{BC})}{2[\ov{AB}^2\ov{CA}^2-(\ov{AB}\cdot
  \ov{CA})^2]}\ov{CA}\\&=-\frac{\ov{CA}^2(\ov{AB}\cdot
  \ov{BC})}{2(\ov{CA}\times
  \ov{AB})^2}\ov{AB}-\frac{\ov{AB}^2(\ov{CA}\cdot
  \ov{BC})}{2(\ov{CA}\times \ov{AB})^2}\ov{CA}.
\end{align*}
\end{proof}
\end{document}
