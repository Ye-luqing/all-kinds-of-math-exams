\documentclass[a4paper]{article}
\usepackage{amsmath,amsfonts,amsthm,amssymb}
\usepackage{bm}
\usepackage{hyperref}
\usepackage{geometry}
\usepackage{yhmath}
\usepackage{pstricks-add}
\usepackage{framed,mdframed}
\usepackage{graphicx,color} 
\usepackage{mathrsfs,xcolor} 
\usepackage[all]{xy}
\usepackage{fancybox} 
\usepackage{xeCJK}
\newtheorem*{theo}{定理}
\newtheorem*{exe}{题目}
\newtheorem*{rem}{评论}
\newtheorem*{lemma}{引理}
\newtheorem*{coro}{推论}
\newtheorem*{exa}{例}
\newenvironment{corollary}
{\bigskip\begin{mdframed}\begin{coro}}
    {\end{coro}\end{mdframed}\bigskip}
\newenvironment{theorem}
{\bigskip\begin{mdframed}\begin{theo}}
    {\end{theo}\end{mdframed}\bigskip}
\newenvironment{exercise}
{\bigskip\begin{mdframed}\begin{exe}}
    {\end{exe}\end{mdframed}\bigskip}
\newenvironment{example}
{\bigskip\begin{mdframed}\begin{exa}}
    {\end{exa}\end{mdframed}\bigskip}
\newenvironment{remark}
{\bigskip\begin{mdframed}\begin{rem}}
    {\end{rem}\end{mdframed}\bigskip}
\geometry{left=2.5cm,right=2.5cm,top=2.5cm,bottom=2.5cm}
\setCJKmainfont[BoldFont=SimHei]{SimSun}
\renewcommand{\today}{\number\year 年 \number\month 月 \number\day 日}
\newcommand{\D}{\displaystyle}\newcommand{\ri}{\Rightarrow}
\newcommand{\ds}{\displaystyle} \renewcommand{\ni}{\noindent}
\newcommand{\ov}{\overrightarrow}
\newcommand{\pa}{\partial} \newcommand{\Om}{\Omega}
\newcommand{\om}{\omega} \newcommand{\sik}{\sum_{i=1}^k}
\newcommand{\vov}{\Vert\omega\Vert} \newcommand{\Umy}{U_{\mu_i,y^i}}
\newcommand{\lamns}{\lambda_n^{^{\scriptstyle\sigma}}}
\newcommand{\chiomn}{\chi_{_{\Omega_n}}}
\newcommand{\ullim}{\underline{\lim}} \newcommand{\bsy}{\boldsymbol}
\newcommand{\mvb}{\mathversion{bold}} \newcommand{\la}{\lambda}
\newcommand{\La}{\Lambda} \newcommand{\va}{\varepsilon}
\newcommand{\be}{\beta} \newcommand{\al}{\alpha}
\newcommand{\dis}{\displaystyle} \newcommand{\R}{{\mathbb R}}
\newcommand{\N}{{\mathbb N}} \newcommand{\cF}{{\mathcal F}}
\newcommand{\gB}{{\mathfrak B}} \newcommand{\eps}{\epsilon}
\renewcommand\refname{参考文献}\renewcommand\figurename{图}
\usepackage[]{caption2} 
\renewcommand{\captionlabeldelim}{}
\setlength\parindent{0pt}
\begin{document}
\title{\huge{\bf{利用向量法证明三角形面积的海伦公式}}} \author{\small{叶卢庆\footnote{叶卢庆(1992---),男,杭州师范大学理学院数学与应用数学专业本科在读,E-mail:yeluqingmathematics@gmail.com}}}
\maketitle
\begin{exercise}
利用向量法证明关于三角形面积的海伦公式:
$$
S^2=p(p-a)(p-b)(p-c).
$$
其中$a,b,c$表示三角形的边长,$S$代表三角形的面积,$p=\frac{a+b+c}{2}$.
\end{exercise}
\begin{proof}[\textbf{证明}]
如图\eqref{fig:1}所示,
\begin{align}
  S^2&=\frac{1}{4}|\ov{AB}\times \ov{AC}|^2
\\&=\frac{1}{4}(\ov{AB}\times \ov{AC})\cdot
(\ov{AB}\times
\ov{AC})\\&=\frac{1}{4}(|\ov{AB}|^2|\ov{AC}|^2-(\ov{AB}\cdot\ov{AC})^2)
\\&=\frac{1}{4}(|\ov{AB}|^2|\ov{AC}|^2-[\frac{1}{2}[\ov{AB}^2+\ov{AC}^2-(\ov{AB}-\ov{AC})^2]]^{2})
\\&=\frac{c^2b^2-(\frac{c^2+b^2-a^2}{2})^2}{4}\\&=p(p-a)(p-b)(p-c).
\end{align}
  \begin{figure}[h]
\newrgbcolor{zzttqq}{0.6 0.2 0.}
\psset{xunit=1.0cm,yunit=1.0cm,algebraic=true,dimen=middle,dotstyle=o,dotsize=3pt 0,linewidth=0.8pt,arrowsize=3pt 2,arrowinset=0.25}
\begin{pspicture*}(-1.3,-5.82)(22.8,6.3)
\pspolygon[linecolor=zzttqq,fillcolor=zzttqq,fillstyle=solid,opacity=0.1](6.98,4.6)(3.04,-1.62)(13.58,-2.46)
\psline[linecolor=zzttqq](6.98,4.6)(3.04,-1.62)
\psline[linecolor=zzttqq](3.04,-1.62)(13.58,-2.46)
\psline[linecolor=zzttqq](13.58,-2.46)(6.98,4.6)
\begin{scriptsize}
\psdots[dotstyle=*,linecolor=blue](6.98,4.6)
\rput[bl](7.06,4.72){\blue{$A$}}
\psdots[dotstyle=*,linecolor=blue](3.04,-1.62)
\rput[bl](3.12,-1.5){\blue{$B$}}
\psdots[dotstyle=*,linecolor=blue](13.58,-2.46)
\rput[bl](13.66,-2.34){\blue{$C$}}
\rput[bl](4.72,1.68){\zzttqq{$c$}}
\rput[bl](8.28,-2.34){\zzttqq{$a$}}
\rput[bl](10.5,1.3){\zzttqq{$b$}}
\end{scriptsize}
\end{pspicture*}
    \caption{}
    \label{fig:1}
  \end{figure}
\end{proof}
\end{document}
