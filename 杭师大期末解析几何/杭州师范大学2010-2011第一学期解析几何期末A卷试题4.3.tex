\documentclass[a4paper]{article}
\usepackage{amsmath,amsfonts,amsthm,amssymb}
\usepackage{bm}
\usepackage{hyperref}
\usepackage{geometry}
\usepackage{yhmath}
\usepackage{pstricks-add}
\usepackage{framed,mdframed}
\usepackage{graphicx,color} 
\usepackage{mathrsfs,xcolor} 
\usepackage[all]{xy}
\usepackage{fancybox} 
\usepackage{xeCJK}
\newtheorem*{theo}{定理}
\newtheorem*{exe}{题目}
\newtheorem*{rem}{评论}
\newtheorem*{lemma}{引理}
\newtheorem*{coro}{推论}
\newtheorem*{exa}{例}
\newenvironment{corollary}
{\bigskip\begin{mdframed}\begin{coro}}
    {\end{coro}\end{mdframed}\bigskip}
\newenvironment{theorem}
{\bigskip\begin{mdframed}\begin{theo}}
    {\end{theo}\end{mdframed}\bigskip}
\newenvironment{exercise}
{\bigskip\begin{mdframed}\begin{exe}}
    {\end{exe}\end{mdframed}\bigskip}
\newenvironment{example}
{\bigskip\begin{mdframed}\begin{exa}}
    {\end{exa}\end{mdframed}\bigskip}
\newenvironment{remark}
{\bigskip\begin{mdframed}\begin{rem}}
    {\end{rem}\end{mdframed}\bigskip}
\geometry{left=2.5cm,right=2.5cm,top=2.5cm,bottom=2.5cm}
\setCJKmainfont[BoldFont=SimHei]{SimSun}
\renewcommand{\today}{\number\year 年 \number\month 月 \number\day 日}
\newcommand{\D}{\displaystyle}\newcommand{\ri}{\Rightarrow}
\newcommand{\ds}{\displaystyle} \renewcommand{\ni}{\noindent}
\newcommand{\ov}{\overrightarrow}
\newcommand{\pa}{\partial} \newcommand{\Om}{\Omega}
\newcommand{\om}{\omega} \newcommand{\sik}{\sum_{i=1}^k}
\newcommand{\vov}{\Vert\omega\Vert} \newcommand{\Umy}{U_{\mu_i,y^i}}
\newcommand{\lamns}{\lambda_n^{^{\scriptstyle\sigma}}}
\newcommand{\chiomn}{\chi_{_{\Omega_n}}}
\newcommand{\ullim}{\underline{\lim}} \newcommand{\bsy}{\boldsymbol}
\newcommand{\mvb}{\mathversion{bold}} \newcommand{\la}{\lambda}
\newcommand{\La}{\Lambda} \newcommand{\va}{\varepsilon}
\newcommand{\be}{\beta} \newcommand{\al}{\alpha}
\newcommand{\dis}{\displaystyle} \newcommand{\R}{{\mathbb R}}
\newcommand{\N}{{\mathbb N}} \newcommand{\cF}{{\mathcal F}}
\newcommand{\gB}{{\mathfrak B}} \newcommand{\eps}{\epsilon}
\renewcommand\refname{参考文献}\renewcommand\figurename{图}
\usepackage[]{caption2} 
\renewcommand{\captionlabeldelim}{}
\setlength\parindent{0pt}
\begin{document}
\title{\huge{\bf{题目4.3}}} \author{\small{叶卢庆\footnote{叶卢庆
      (1992---),男,杭州师范大学理学院数学与应用数学专业本科在
      读,E-mail:yeluqingmathematics@gmail.com}}}
\maketitle
\begin{exercise}
求直线
\begin{equation}\label{eq:1}
\frac{x}{2}=\frac{y}{1}=\frac{z-1}{1}
\end{equation}
绕直线
\begin{equation}\label{eq:2}
\frac{x}{1}=\frac{y}{-1}=\frac{z-1}{2}
\end{equation}
旋转所得的圆锥面方程.
\end{exercise}
此题易得有简易解法,但是下面我呈现比较比较繁难的解法.
\begin{proof}[\textbf{解}]
设旋转所得的曲面上任意一点为$P=(x,y,z)$,则在直线\eqref{eq:1}上存在点
$P_{0}=(x_0,y_0,z_0)$位于曲面上,且$P$和$P_0$到直线\eqref{eq:2}的距离相
等.点$P$到直线\eqref{eq:2}的距离为
$$
\frac{\left|\left[(0,0,1)-(x,y,z)\right]\times (t,-t,2t)\right|}{\sqrt{t^2+(-t)^2+(2t)^2}},
$$
因此
\begin{equation}
  \label{eq:3}
  \frac{\left|\left[(0,0,1)-(x,y,z)\right]\times (t,-t,2t)\right|}{\sqrt{t^2+(-t)^2+(2t)^2}}=\frac{\left|\left[(0,0,1)-(x_0,y_0,z_0)\right]\times (t,-t,2t)\right|}{\sqrt{t^2+(-t)^2+(2t)^2}}.
\end{equation}
化简方程\eqref{eq:3}可得
\begin{equation}
  \label{eq:4}
5x^{2}+5y^{2}+2z^2+2xy+4yz-4zx+4x-4y-4z=5x_{0}^{2}+5y_{0}^{2}+2z_{0}^2+2x_{0}y_{0}+4y_{0}z_{0}-4z_{0}x_{0}+4x_{0}-4y_{0}-4z_0.
\end{equation}
另外,我们还有
\begin{equation}
  \label{eq:5}
  \frac{x_0}{2}=\frac{y_0}{1}=\frac{z_0-1}{1},
\end{equation}
以及
\begin{equation}
  \label{eq:6}
  (x-x_0,y-y_0,z-z_0)\cdot (1,-1,2)=0.
\end{equation}
联立方程\eqref{eq:4},\eqref{eq:5},可得
\begin{equation}\label{eq:7}
5x^{2}+5y^{2}+2z^2+2xy+4yz-4zx+4x-4y-4z=27y_0^2-2,
\end{equation}
联立方程\eqref{eq:5},\eqref{eq:6},可得
\begin{equation}
  \label{eq:8}
  x-y+2z=3y_0+2.
\end{equation}
联立方程\eqref{eq:7},\eqref{eq:8},可得
\begin{equation}
  \label{eq:9}
  5x^{2}+5y^{2}+2z^2+2xy+4yz-4zx+4x-4y-4z=3(x-y+2z-2)^2-2,
\end{equation}
即
$$
x^2+y^2-5z^2+4xy+8yz-8zx+8x-8y+10z-5=0.
$$
这就是欲得锥面方程.
\end{proof}
\end{document}
