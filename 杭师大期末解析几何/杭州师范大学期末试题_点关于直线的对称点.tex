\documentclass[a4paper]{article}
\usepackage{amsmath,amsfonts,amsthm,amssymb}
\usepackage{bm}
\usepackage{draftwatermark,euler}
\SetWatermarkText{http://blog.sciencenet.cn/u/Yaleking}%设置水印文字
\SetWatermarkLightness{0.8}%设置水印亮度
\SetWatermarkScale{0.35}%设置水印大小
\usepackage{hyperref}
\usepackage{geometry}
\usepackage{yhmath}
\usepackage{pstricks-add}
\usepackage{framed,mdframed}
\usepackage{graphicx,color} 
\usepackage{mathrsfs,xcolor} 
\usepackage[all]{xy}
\usepackage{fancybox} 
\usepackage{xeCJK}
\newtheorem*{theo}{定理}
\newtheorem*{exe}{题目}
\newtheorem*{rem}{评论}
\newmdtheoremenv{lemma}{引理}
\newmdtheoremenv{corollary}{推论}
\newmdtheoremenv{example}{例}
\newenvironment{theorem}
{\bigskip\begin{mdframed}\begin{theo}}
    {\end{theo}\end{mdframed}\bigskip}
\newenvironment{exercise}
{\bigskip\begin{mdframed}\begin{exe}}
    {\end{exe}\end{mdframed}\bigskip}
\geometry{left=2.5cm,right=2.5cm,top=2.5cm,bottom=2.5cm}
\setCJKmainfont[BoldFont=SimHei]{SimSun}
\renewcommand{\today}{\number\year 年 \number\month 月 \number\day 日}
\newcommand{\D}{\displaystyle}\newcommand{\ri}{\Rightarrow}
\newcommand{\ds}{\displaystyle} \renewcommand{\ni}{\noindent}
\newcommand{\ov}{\overrightarrow}
\newcommand{\pa}{\partial} \newcommand{\Om}{\Omega}
\newcommand{\om}{\omega} \newcommand{\sik}{\sum_{i=1}^k}
\newcommand{\vov}{\Vert\omega\Vert} \newcommand{\Umy}{U_{\mu_i,y^i}}
\newcommand{\lamns}{\lambda_n^{^{\scriptstyle\sigma}}}
\newcommand{\chiomn}{\chi_{_{\Omega_n}}}
\newcommand{\ullim}{\underline{\lim}} \newcommand{\bsy}{\boldsymbol}
\newcommand{\mvb}{\mathversion{bold}} \newcommand{\la}{\lambda}
\newcommand{\La}{\Lambda} \newcommand{\va}{\varepsilon}
\newcommand{\be}{\beta} \newcommand{\al}{\alpha}
\newcommand{\dis}{\displaystyle} \newcommand{\R}{{\mathbb R}}
\newcommand{\N}{{\mathbb N}} \newcommand{\cF}{{\mathcal F}}
\newcommand{\gB}{{\mathfrak B}} \newcommand{\eps}{\epsilon}
\renewcommand\refname{参考文献}\renewcommand\figurename{图}
\usepackage[]{caption2} 
\renewcommand{\captionlabeldelim}{}
\setlength\parindent{0pt}
\begin{document}
\title{\huge{\bf{杭州师范大学一道解析几何期末试题}}} \author{\small{叶卢庆\footnote{叶卢庆(1992---),男,杭州师范大学理学院数学与应用数学专业本科在读,E-mail:yeluqingmathematics@gmail.com}}}
\maketitle
\begin{exercise}
  求点$A(a,b,c)$关于直线
  $\frac{x-x_0}{\cos\alpha}=\frac{y-y_0}{\cos\beta}=\frac{z-z_0}{\cos\gamma}$($\alpha,\beta,\gamma$
  为直线的方向角)的对称点的坐标.
\end{exercise}
\begin{proof}[\textbf{解}]
首先,我们要在直线上选取一点$P=(x',y',z')$,使得$\ov{AP}\cdot
(\cos\alpha,\cos\beta,\cos\gamma)=0$.也就是说,
$$
\begin{cases}
  \frac{x'-x_0}{\cos\alpha}=\frac{y'-y_{0}}{\cos\beta}=\frac{z'-z_0}{\cos\gamma}=t,\\
(x'-a)\cos\alpha+(y'-b)\cos\beta+(z'-c)\cos\gamma=0.
\end{cases}
$$
解得
$$
t(\cos^2\alpha+\cos^2\beta+\cos^2\gamma)=(a-x_0)\cos\alpha+(b-y_0)\cos\beta+(c-z_0)\cos\gamma,
$$
由于$\cos^2\alpha+\cos^2\beta+\cos^2\gamma=1$,因此
$$
t=(a-x_0)\cos\alpha+(b-y_0)\cos\beta+(c-z_0)\cos\gamma.
$$
因此得到
$$
\begin{cases}
  x'=t\cos\alpha+x_0=(a-x_0)\cos^{2}\alpha+(b-y_0)\cos\beta\cos\alpha+(c-z_0)\cos\gamma\cos\alpha+x_0,\\
  y'=t\cos\beta+y_0=(a-x_0)\cos\alpha\cos\beta+(b-y_0)\cos^{2}\beta+(c-z_0)\cos\gamma\cos\beta+y_0,\\
z'=t\cos\gamma+z_0=(a-x_0)\cos\alpha\cos\gamma+(b-y_0)\cos\beta\cos\gamma+(c-z_0)\cos^2\gamma+z_0.
\end{cases}
$$
对称点坐标设为$(a',b',c')$.易得,
\begin{align*}
(a',b',c')-(a,b,c)=2\ov{AP}.
\end{align*}
这样就能解出$a',b',c'$,最终得到对称点坐标.
\end{proof}
\end{document}
