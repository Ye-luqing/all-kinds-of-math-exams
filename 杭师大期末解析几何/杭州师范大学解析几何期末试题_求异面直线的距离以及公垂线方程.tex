\documentclass[a4paper]{article}
\usepackage{amsmath,amsfonts,amsthm,amssymb}
\usepackage{bm}
\usepackage{hyperref}
\usepackage{geometry}
\usepackage{yhmath}
\usepackage{pstricks-add}
\usepackage{framed,mdframed}
\usepackage{graphicx,color} 
\usepackage{mathrsfs,xcolor} 
\usepackage[all]{xy}
\usepackage{fancybox} 
\usepackage{xeCJK}
\newtheorem*{theo}{定理}
\newtheorem*{exe}{题目}
\newtheorem*{rem}{评论}
\newtheorem*{lemma}{引理}
\newtheorem*{coro}{推论}
\newtheorem*{exa}{例}
\newenvironment{corollary}
{\bigskip\begin{mdframed}\begin{coro}}
    {\end{coro}\end{mdframed}\bigskip}
\newenvironment{theorem}
{\bigskip\begin{mdframed}\begin{theo}}
    {\end{theo}\end{mdframed}\bigskip}
\newenvironment{exercise}
{\bigskip\begin{mdframed}\begin{exe}}
    {\end{exe}\end{mdframed}\bigskip}
\newenvironment{example}
{\bigskip\begin{mdframed}\begin{exa}}
    {\end{exa}\end{mdframed}\bigskip}
\newenvironment{remark}
{\bigskip\begin{mdframed}\begin{rem}}
    {\end{rem}\end{mdframed}\bigskip}
\geometry{left=2.5cm,right=2.5cm,top=2.5cm,bottom=2.5cm}
\setCJKmainfont[BoldFont=SimHei]{SimSun}
\renewcommand{\today}{\number\year 年 \number\month 月 \number\day 日}
\newcommand{\D}{\displaystyle}\newcommand{\ri}{\Rightarrow}
\newcommand{\ds}{\displaystyle} \renewcommand{\ni}{\noindent}
\newcommand{\ov}{\overrightarrow}
\newcommand{\pa}{\partial} \newcommand{\Om}{\Omega}
\newcommand{\om}{\omega} \newcommand{\sik}{\sum_{i=1}^k}
\newcommand{\vov}{\Vert\omega\Vert} \newcommand{\Umy}{U_{\mu_i,y^i}}
\newcommand{\lamns}{\lambda_n^{^{\scriptstyle\sigma}}}
\newcommand{\chiomn}{\chi_{_{\Omega_n}}}
\newcommand{\ullim}{\underline{\lim}} \newcommand{\bsy}{\boldsymbol}
\newcommand{\mvb}{\mathversion{bold}} \newcommand{\la}{\lambda}
\newcommand{\La}{\Lambda} \newcommand{\va}{\varepsilon}
\newcommand{\be}{\beta} \newcommand{\al}{\alpha}
\newcommand{\dis}{\displaystyle} \newcommand{\R}{{\mathbb R}}
\newcommand{\N}{{\mathbb N}} \newcommand{\cF}{{\mathcal F}}
\newcommand{\gB}{{\mathfrak B}} \newcommand{\eps}{\epsilon}
\renewcommand\refname{参考文献}\renewcommand\figurename{图}
\usepackage[]{caption2} 
\renewcommand{\captionlabeldelim}{}
\setlength\parindent{0pt}
\begin{document}
\begin{exercise}
  已知两异面直线
  \begin{equation}
    \label{eq:1}
    l_1:\frac{x-3}{2}=\frac{y}{1}=\frac{z-1}{0}
  \end{equation}
与
\begin{equation}
  \label{eq:2}
  l_2:\frac{x+1}{1}=\frac{y-2}{0}=\frac{z}{1}
\end{equation}
试求$l_1,l_2$的距离与它们的公垂线方程.
\end{exercise}
\begin{proof}[\textbf{解}]
直线\eqref{eq:1}上有点$P=(3,0,1)$,直线\eqref{eq:2}上有点
$Q=(-1,2,0)$.且与两直线的方向向量都垂直的一个向量为
$$
\mathbf{a}=(2,1,0)\times (1,0,1)=(1,-2,-1).
$$
因此两直线的距离为
$$
\ov{PQ}\cdot \mathbf{a}\frac{1}{|\ov{a}|}=\frac{-7 \sqrt{6}}{6}.
$$
下面我们来求公垂线方程.我们先求公垂线与直线\eqref{eq:1}形成的平面$p_{1}$的方
程.易得$p_1$的法向量为
$$
\mathbf{a}\times (2,1,0)=(1,-2,5),
$$
因此可设$p_1$为
\begin{equation}
  \label{eq:3}
  x-2y+5z+d_1=0.
\end{equation}
且平面\eqref{eq:3}经过点$(3,0,1)$,因此$d_1=-8$.可见,平面\eqref{eq:3}的
方程为
\begin{equation}
  \label{eq:4}
  x-2y+5z-8=0.
\end{equation}
再求公垂线与直线\eqref{eq:2}形成的平面$p_2$的方程.易得$p_2$的法向量为
$$
\mathbf{a}\times (1,0,1)=(-2,4,2).
$$
因此平面$p_2$可以设为
\begin{equation}
  \label{eq:5}
  -x+2y+z+d_2=0.
\end{equation}
由于平面$p_2$经过点$(-1,2,0)$,因此$d_2=-5$.因此平面$p_2$为
\begin{equation}
  \label{eq:6}
  -x+2y+z-5=0.
\end{equation}
可见,公垂线的方程为
\begin{equation}
  \label{eq:7}
  \begin{cases}
    6z-13=0\\
x-2y+5z-8=0
  \end{cases}
\end{equation}
\end{proof}
\end{document}
