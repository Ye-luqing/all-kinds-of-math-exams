\documentclass[a4paper]{article}
\usepackage{amsmath,amsfonts,amsthm,amssymb} \usepackage{bm}
\usepackage{draftwatermark,euler}
\SetWatermarkText{http://blog.sciencenet.cn/u/Yaleking}%设置水印文字
\SetWatermarkLightness{0.8}%设置水印亮度
\SetWatermarkScale{0.35}%设置水印大小
\usepackage{hyperref} \usepackage{geometry} \usepackage{yhmath}
\usepackage{pstricks-add} \usepackage{framed,mdframed}
\usepackage{graphicx,color} \usepackage{mathrsfs,xcolor}
\usepackage[all]{xy} \usepackage{fancybox} \usepackage{xeCJK}
\newtheorem*{theo}{定理} 
\newtheorem*{exe}{题目}
\newenvironment{theorem}
{\bigskip\begin{mdframed}\begin{theo}}
    {\end{theo}\end{mdframed}\bigskip} 
\newenvironment{exercise}
{\bigskip\begin{mdframed}\begin{exe}}
    {\end{exe}\end{mdframed}\bigskip}
\geometry{left=2.5cm,right=2.5cm,top=2.5cm,bottom=2.5cm}
\setCJKmainfont[BoldFont=SimHei]{SimSun}
\numberwithin{equation}{section}
\setlength\parindent{0pt}
\newcommand{\D}{\displaystyle}\newcommand{\ri}{\Rightarrow}
\newcommand{\ds}{\displaystyle} \renewcommand{\ni}{\noindent}
\newcommand{\pa}{\partial} \newcommand{\Om}{\Omega}
\newcommand{\om}{\omega} \newcommand{\sik}{\sum_{i=1}^k}
\newcommand{\vov}{\Vert\omega\Vert} \newcommand{\Umy}{U_{\mu_i,y^i}}
\newcommand{\lamns}{\lambda_n^{^{\scriptstyle\sigma}}}
\newcommand{\chiomn}{\chi_{_{\Omega_n}}}
\newcommand{\ullim}{\underline{\lim}}
\newcommand{\mvb}{\mathversion{bold}} \newcommand{\la}{\lambda}
\newcommand{\La}{\Lambda} \newcommand{\va}{\varepsilon}
\newcommand{\be}{\beta}
\newcommand{\dis}{\displaystyle} \newcommand{\R}{{\mathbb R}}
\renewcommand{\today}{\number\year 年 \number\month 月 \number\day 日}
\newcommand{\N}{{\mathbb N}} \newcommand{\cF}{{\mathcal F}}
\newcommand{\gB}{{\mathfrak B}} \newcommand{\eps}{\epsilon}
\renewcommand\refname{参考文献}\renewcommand\figurename{图}
\usepackage[]{caption2} \renewcommand{\captionlabeldelim}{}
\begin{document}
\title{{\bf{2014中科大考研《线性代数与解析几何》之解析几何解
      答\footnote{本解答作为交给解析几何赵老师的第三份作业.}}}} \author{\small{叶卢庆
    \footnote{叶卢庆(1992-),男,杭州师范大学理学院数学与应用数学专业大
      四.学号:1002011005.E-mail:yeluqingmathematics@gmail.com}}}
\maketitle
\begin{exercise}[1]
原点到直线$x+1=y+2=z+3$的距离为?  
\end{exercise}
\begin{proof}[\textbf{解}]
已知直线上的任意一点$(t,t-1,t-2)$,求
$$
\sqrt{t^2+(t-1)^2+(t-2)^2}
$$
的最小值.$t^2+(t-1)^2+(t-2)^2=3t^2-6t+5$.易得当$t=1$时,$3t^2-6t+5$取得
最小值$2$.因此原点到直线的距离为$\sqrt{2}$.
\end{proof}
\begin{exercise}[2]
设点$P(1,2,3)$与原点关于平面$\pi$对称,则$\pi$的方程为?  
\end{exercise}
\begin{proof}[\textbf{解}]
显然,平面$\pi$的一个法向量为$(1,2,3)$,且平面$\pi$通过点
$(\frac{1}{2},1,\frac{3}{2})$,因此平面$\pi$的方程为
$$
x+2y+3z-7=0.
$$
\end{proof}
\begin{exercise}[3]
椭圆$x^2+xy+y^2=1$的离心率为?  
\end{exercise}
\begin{proof}[\textbf{解}]
转轴.令
$$
\begin{pmatrix}
  x\\
y
\end{pmatrix}=
\begin{pmatrix}
  \frac{1}{\sqrt{2}}&\frac{1}{\sqrt{2}}\\
-\frac{1}{\sqrt{2}}&\frac{1}{\sqrt{2}}
\end{pmatrix}
\begin{pmatrix}
  x'\\
y'
\end{pmatrix}.
$$
代入椭圆方程,可得
$$
\frac{x'^2}{(\sqrt{2})^{2}}+\frac{y'^2}{(\sqrt{\frac{2}{3}})^2}=1.
$$
因此椭圆的离心率为$\frac{\sqrt{6}}{3}$.
\end{proof}
\end{document}