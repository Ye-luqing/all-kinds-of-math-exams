\documentclass[a3paper]{article}
\usepackage{amsmath,amsfonts,amsthm,amssymb} \usepackage{bm}
\usepackage{draftwatermark,euler}
\SetWatermarkText{http://blog.sciencenet.cn/u/Yaleking}%设置水印文字
\SetWatermarkLightness{0.8}%设置水印亮度
\SetWatermarkScale{0.35}%设置水印大小
\usepackage{hyperref} \usepackage{geometry} \usepackage{yhmath}
\usepackage{pstricks-add} \usepackage{framed,mdframed}
\usepackage{graphicx,color} \usepackage{mathrsfs,xcolor}
\usepackage[all]{xy} \usepackage{fancybox} \usepackage{xeCJK}
\newtheorem*{theo}{定理} 
\newtheorem*{exe}{题目}
\newenvironment{theorem}
{\bigskip\begin{mdframed}\begin{theo}}
    {\end{theo}\end{mdframed}\bigskip} 
\newenvironment{exercise}
{\bigskip\begin{mdframed}\begin{exe}}
    {\end{exe}\end{mdframed}\bigskip}
\geometry{left=2.5cm,right=2.5cm,top=2.5cm,bottom=2.5cm}
\setCJKmainfont[BoldFont=SimHei]{SimSun}
\numberwithin{equation}{section}
\setlength\parindent{0pt}
\newcommand{\D}{\displaystyle}\newcommand{\ri}{\Rightarrow}
\newcommand{\ds}{\displaystyle} \renewcommand{\ni}{\noindent}
\newcommand{\pa}{\partial} \newcommand{\Om}{\Omega}
\newcommand{\om}{\omega} \newcommand{\sik}{\sum_{i=1}^k}
\newcommand{\vov}{\Vert\omega\Vert} \newcommand{\Umy}{U_{\mu_i,y^i}}
\newcommand{\lamns}{\lambda_n^{^{\scriptstyle\sigma}}}
\newcommand{\chiomn}{\chi_{_{\Omega_n}}}
\newcommand{\ullim}{\underline{\lim}} \newcommand{\ov}{\overrightarrow}
\newcommand{\mvb}{\mathversion{bold}} \newcommand{\la}{\lambda}
\newcommand{\La}{\Lambda} \newcommand{\va}{\varepsilon}
\newcommand{\be}{\beta} \newcommand{\al}{\alpha}
\newcommand{\dis}{\displaystyle} \newcommand{\R}{{\mathbb R}}
\renewcommand{\today}{\number\year 年 \number\month 月 \number\day 日}
\newcommand{\N}{{\mathbb N}} \newcommand{\cF}{{\mathcal F}}
\newcommand{\gB}{{\mathfrak B}} \newcommand{\eps}{\epsilon}
\renewcommand\refname{参考文献}\renewcommand\figurename{图}
\usepackage[]{caption2} \renewcommand{\captionlabeldelim}{}
\begin{document}
\title{{\bf{2010中科大考研《线性代数与解析几何》之解析几何题目解
      答\footnote{本
        解答作为交给解析几何赵老师的第三份作业.}}}} \author{\small{叶卢庆
    \footnote{叶卢庆(1992-),男,杭州师范大学理学院数学与应用数学专业大
      四.学号:1002011005.E-mail:yeluqingmathematics@gmail.com}}}
\maketitle
\begin{exercise}[1]
二次曲线$x^2-4xy+y^2+10x-10y+21=0$的类型是?通过转轴去掉其交叉项的转角
角度为?  
\end{exercise}
\begin{proof}[\textbf{解}]
令
$$
\begin{cases}
  x=x'\cos\alpha-y'\sin\alpha,\\
  y=x'\sin\alpha+y'\cos\alpha.
\end{cases}
$$
代入方程,可得
$$
(x'\cos\alpha-y'\sin\alpha)^2-4(x'\cos\alpha-y'\sin\alpha)(x'\sin\alpha+y'\cos\alpha)+(x'\sin\alpha+y'\cos\alpha)^{2}+10(x'\cos\alpha-y'\sin\alpha)-10(x'\sin\alpha+y'\cos\alpha)+21=0.
$$
交叉项$x'y'$前面的系数为$-4\cos 2\alpha$,令其等于零,解得$\alpha$可以为
$\frac{\pi}{4}$.于是通过转轴去掉交叉项的角度为$\frac{\pi}{4}$.此时,方
程化简为
$$
(y'-\frac{5 \sqrt{2}}{3})^2-\frac{x^2}{3}=-\frac{13}{9}.
$$
可见,二次曲线类型为双曲线.
\end{proof}
\begin{exercise}[2]
以曲线$
\begin{cases}
  y=x^2\\
z=2
\end{cases}
$ 为准线,原点为顶点的锥面方程为? 
\end{exercise}
\begin{proof}[\textbf{解}]
设$(x,y,z)$是锥面上的任意一点(但不是原点).则必定存在$t\in
\mathbf{R},t\neq 0$,使得
$$
t(x,y,z)=(m,m^2,2)
$$
于是,$2x^2=yz$.这就是锥面方程.
\end{proof}
\begin{exercise}[3]
  \begin{itemize}
  \item 以$xOy$平面上的曲线$f(x,y)=0$绕$x$轴旋转所得的旋转面的方程为?  
\item 如果曲线方程是$x^2-y^2-1=0$,则由此得到的曲面类型为?
  \end{itemize}
\end{exercise}
\begin{proof}[\textbf{解}]
  \begin{itemize}
  \item 设点$(x_0,y_0,0)$满足$f(x_0,y_0)=0$.然后开始计时,经过时间$t$后,
    点$(x_0,y_0)$运动到$(x_0,y_1,z_1)$.则我们有
$$
y_0^2=y_1^2+z_1^2.
$$
于是旋转面的方程为
$$
f(x,\pm \sqrt{y^2+z^2})=0.
$$
\item 单叶双曲面.
  \end{itemize}
\end{proof}
\begin{exercise}[解答题第三题]
  设空间上有直线$l_1:\frac{x-1}{3}=\frac{y}{1}=\frac{z}{0}$和
  $l_2:(x,y,z)=(3+2t,t,3t-3)$.设平面$\pi$与直线$l_1,l_2$平行,且$\pi$与
  $l_1$的距离为$\sqrt{91}$,求$\pi$的方程.
\end{exercise}
\begin{proof}[\textbf{解}]
直线$l_1$的方向向量为$(3,1,0)=3i+j$,$l_2$的方向向量为
$(2,1,3)=2i+j+3k$.可见,平面$\pi$的一个法向量为
$$
(3i+j)\times (2i+j+3k)=3i-9j+k=(3,-9,1).
$$
于是,平面$\pi$可以设为
$$
3x-9y+z+d=0.
$$
由于直线$l_1$上有点$(1,0,0)$,该点到平面$\pi$的距离为$\sqrt{91}$,也就是
说,已知$3(x-1)-9y+z=-d-3$,且
$$
(x-1)^2+y^2+z^2
$$
的最小值为$91$.根据Cauchy不等式,
$$
-d-3=3(x-1)-9y+z\leq \sqrt{3^2+9^2+1^2}\sqrt{(x-1)^2+y^2+z^2}.
$$
且等号能取到.可见,$-d-3=91$,于是,$d=-94$.于是,平面$\pi$的方程是
$$
3x-9y+z-94=0.
$$
\end{proof}
\end{document}