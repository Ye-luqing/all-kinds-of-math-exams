\documentclass[a4paper]{article}
\usepackage{amsmath,amsfonts,amsthm,amssymb} \usepackage{bm}
\usepackage{draftwatermark,euler}
\SetWatermarkText{http://blog.sciencenet.cn/u/Yaleking}%设置水印文字
\SetWatermarkLightness{0.8}%设置水印亮度
\SetWatermarkScale{0.35}%设置水印大小
\usepackage{hyperref} \usepackage{geometry} \usepackage{yhmath}
\usepackage{pstricks-add} \usepackage{framed,mdframed}
\usepackage{graphicx,color} \usepackage{mathrsfs,xcolor}
\usepackage[all]{xy} \usepackage{fancybox} \usepackage{xeCJK}
\newtheorem*{theo}{定理} 
\newtheorem*{exe}{题目}
\newenvironment{theorem}
{\bigskip\begin{mdframed}\begin{theo}}
    {\end{theo}\end{mdframed}\bigskip} 
\newenvironment{exercise}
{\bigskip\begin{mdframed}\begin{exe}}
    {\end{exe}\end{mdframed}\bigskip}
\geometry{left=2.5cm,right=2.5cm,top=2.5cm,bottom=2.5cm}
\setCJKmainfont[BoldFont=SimHei]{SimSun}
\numberwithin{equation}{section}
\setlength\parindent{0pt}
\newcommand{\D}{\displaystyle}\newcommand{\ri}{\Rightarrow}
\newcommand{\ds}{\displaystyle} \renewcommand{\ni}{\noindent}
\newcommand{\pa}{\partial} \newcommand{\Om}{\Omega}
\newcommand{\om}{\omega} \newcommand{\sik}{\sum_{i=1}^k}
\newcommand{\vov}{\Vert\omega\Vert} \newcommand{\Umy}{U_{\mu_i,y^i}}
\newcommand{\lamns}{\lambda_n^{^{\scriptstyle\sigma}}}
\newcommand{\chiomn}{\chi_{_{\Omega_n}}}
\newcommand{\ullim}{\underline{\lim}}
\newcommand{\mvb}{\mathversion{bold}} \newcommand{\la}{\lambda}
\newcommand{\La}{\Lambda} \newcommand{\va}{\varepsilon}
\newcommand{\be}{\beta} \newcommand{\ov}{\overrightarrow}
\newcommand{\dis}{\displaystyle} \newcommand{\R}{{\mathbb R}}
\renewcommand{\today}{\number\year 年 \number\month 月 \number\day 日}
\newcommand{\N}{{\mathbb N}} \newcommand{\cF}{{\mathcal F}}
\newcommand{\gB}{{\mathfrak B}} \newcommand{\eps}{\epsilon}
\renewcommand\refname{参考文献}\renewcommand\figurename{图}
\usepackage[]{caption2} \renewcommand{\captionlabeldelim}{}
\begin{document}
\title{{\bf{2011中科大考研《线性代数与解析几何》之解析几何解
      答\footnote{本
        解答作为交给解析几何赵老师的第三份作业.}}}} \author{\small{叶卢庆
    \footnote{叶卢庆(1992-),男,杭州师范大学理学院数学与应用数学专业大
      四.学号:1002011005.E-mail:yeluqingmathematics@gmail.com}}}
\maketitle
\begin{exercise}[1]
两平面$z=x+2y$和$z=-2x-y$的夹角等于?  
\end{exercise}
\begin{proof}[\textbf{解}]
平面$x+2y-z=0$的法向量为$\ov{n_1}=(1,2,-1)$.平面$-2x-y-z=0$的法向量为
$\ov{n_2}=(-2,-1,-1)$.
$$
\cos\langle n_1,n_2\rangle=\frac{\ov{n_1}\cdot \ov{n_2}}{|\ov{n_1}||\ov{n_2}|}=-\frac{1}{2}.
$$
所以两平面的夹角为$\frac{\pi}{3}$.  
\end{proof}
\begin{exercise}[2]
点$(0,2,1)$到平面$2x-3y+6z=1$的距离等于?  
\end{exercise}
\begin{proof}[\textbf{解}]
即在限制条件$2x-3y+6z=1$下求
$$
\sqrt{x^2+(y-2)^2+(z-1)^2}
$$
的最小值.根据Cauchy不等式,
$$
[x^2+(y-2)^2+(z-1)^2][2^2+(-3)^2+6^2]\geq (2x-3y+6+6z-6)^2=1.
$$
且等号能取到.可见,
$$
\sqrt{x^2+(y-2)^2+(z-1)^2}\geq \frac{1}{7}.
$$
于是距离就是$\frac{1}{7}$.
\end{proof}
\begin{exercise}[3]
  二次曲面$xy+z^2=1$的曲面类型是?
\end{exercise}
\begin{proof}[\textbf{解}]
  为了消去$xy$项,我们进行转轴.令
$$
\begin{cases}
  x=x'\cos\alpha-y'\sin\alpha\\
y=x'\sin\alpha+y'\cos\alpha\\
z=z'
\end{cases},
$$
令$\alpha=\frac{\pi}{4}$,则二次曲面可以化为
$$
\frac{x'^2}{2}-\frac{y'^2}{2}+z'^2=1,
$$
于是二次曲面的类型是单叶双曲面.
\end{proof}
\begin{exercise}[11]
  设点$A(1,1,-1)$,$B(-1,1,1),C(1,1,1)$,求$\triangle ABC$的外接圆的方程.
\end{exercise}
\begin{proof}[\textbf{解}]
我们先求出经过这三个点的平面方程.易得
$\ov{AB}=(-2,0,2),\ov{AC}=(0,0,2)$.由于
$$
\ov{AB}\times \ov{AC}=(0,4,0),
$$
因此平面的方程为$y=1$.于是易得外接圆的方程为
$$
\begin{cases}
  x^2+z^2=2,\\
y=1
\end{cases}.
$$
\end{proof}
\end{document}