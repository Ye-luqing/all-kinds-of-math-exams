\documentclass[a3paper]{article}
\usepackage{amsmath,amsfonts,amsthm,amssymb} \usepackage{bm}
\usepackage{draftwatermark,euler}
\SetWatermarkText{http://blog.sciencenet.cn/u/Yaleking}%设置水印文字
\SetWatermarkLightness{0.8}%设置水印亮度
\SetWatermarkScale{0.35}%设置水印大小
\usepackage{hyperref} \usepackage{geometry} \usepackage{yhmath}
\usepackage{pstricks-add} \usepackage{framed,mdframed}
\usepackage{graphicx,color} \usepackage{mathrsfs,xcolor}
\usepackage[all]{xy} \usepackage{fancybox} \usepackage{xeCJK}
\newtheorem*{theo}{定理} 
\newtheorem*{exe}{题目}
\newenvironment{theorem}
{\bigskip\begin{mdframed}\begin{theo}}
    {\end{theo}\end{mdframed}\bigskip} 
\newenvironment{exercise}
{\bigskip\begin{mdframed}\begin{exe}}
    {\end{exe}\end{mdframed}\bigskip}
\geometry{left=2.5cm,right=2.5cm,top=2.5cm,bottom=2.5cm}
\setCJKmainfont[BoldFont=SimHei]{SimSun}
\numberwithin{equation}{section}
\setlength\parindent{0pt}
\newcommand{\D}{\displaystyle}\newcommand{\ri}{\Rightarrow}
\newcommand{\ds}{\displaystyle} \renewcommand{\ni}{\noindent}
\newcommand{\pa}{\partial} \newcommand{\Om}{\Omega}
\newcommand{\om}{\omega} \newcommand{\sik}{\sum_{i=1}^k}
\newcommand{\vov}{\Vert\omega\Vert} \newcommand{\Umy}{U_{\mu_i,y^i}}
\newcommand{\lamns}{\lambda_n^{^{\scriptstyle\sigma}}}
\newcommand{\chiomn}{\chi_{_{\Omega_n}}}
\newcommand{\ullim}{\underline{\lim}} \newcommand{\ov}{\overrightarrow}
\newcommand{\mvb}{\mathversion{bold}} \newcommand{\la}{\lambda}
\newcommand{\La}{\Lambda} \newcommand{\va}{\varepsilon}
\newcommand{\be}{\beta} \newcommand{\al}{\alpha}
\newcommand{\dis}{\displaystyle} \newcommand{\R}{{\mathbb R}}
\renewcommand{\today}{\number\year 年 \number\month 月 \number\day 日}
\newcommand{\N}{{\mathbb N}} \newcommand{\cF}{{\mathcal F}}
\newcommand{\gB}{{\mathfrak B}} \newcommand{\eps}{\epsilon}
\renewcommand\refname{参考文献}\renewcommand\figurename{图}
\usepackage[]{caption2} \renewcommand{\captionlabeldelim}{}
\begin{document}
\title{\huge{\bf{2013中科大考研《线性代数与解析几何》之解析几何题目解
      答\footnote{本
        解答作为交给解析几何赵老师的第三份作业.}}}} \author{\small{叶卢庆
    \footnote{叶卢庆(1992-),男,杭州师范大学理学院数学与应用数学专业大
      四.学号:1002011005.E-mail:yeluqingmathematics@gmail.com}}}
\maketitle
\begin{exercise}[1]
求两直线$1-x=2y=3z$与$x=y+2=2z+4$的夹角和距离.
\end{exercise}
\begin{proof}[\textbf{解}]
第一条直线的方程可以化为
$$
\frac{x-1}{1}=\frac{y-0}{-\frac{1}{2}}=\frac{z-0}{-\frac{1}{3}}.
$$
可见,第一条直线通过点$p=(1,0,0)$,且直线的方向向量为
$\ov{n_{1}}=(1,\frac{-1}{2},\frac{-1}{3})=e_{1}-\frac{1}{2}e_{2}-\frac{1}{3}e_{3}$.第二条直线的方程可以化为
$$
\frac{x-0}{1}=\frac{y+2}{1}=\frac{z+2}{\frac{1}{2}}.
$$
可见,第二条直线通过点$q=(0,-2,-2)$,且直线的方向向量为
$\ov{n_2}=(1,1,\frac{1}{2})=e_1+e_2+\frac{1}{2}e_{3}$.设两直线的夹角为$\alpha$,则
$$
\cos\alpha=\frac{|\ov{n_1}\cdot \ov{n_2}|}{|\ov{n_1}||\ov{n_2}|}=\frac{4}{21}.
$$
因此$\alpha=\arccos \frac{4}{21}$.下面我们来求两条直线之间的距离.首先,
我们来求和向量$\ov{n_1},\ov{n_2}$都垂直的向量,这很简单,$\ov{n_1}\times
\ov{n_2}$就满足条件,易得
$$
\ov{n_1}\times \ov{n_2}=(e_1-\frac{1}{2}e_2-\frac{1}{3}e_3)\times (e_1+e_2+\frac{1}{2}e_3)=\frac{1}{12}e_1-\frac{5}{6}e_2+\frac{3}{2}e_3=(\frac{1}{12},\frac{-5}{6},\frac{3}{2}).
$$
两直线之间的距离就等于
$$
\frac{|(\ov{n_1}\times\ov{n_2})\cdot \ov{pq}|}{|\ov{n_1}\times
  \ov{n_2}|}=\frac{17 \sqrt{26}}{104}.
$$
\end{proof}
\begin{exercise}[2]
当实数$a,b,c$满足什么条件时,曲面$z=ax^2+bxy+cy^2$是椭圆抛物面.  
\end{exercise}
\begin{proof}[\textbf{解}]
只需要二次型$ax^2+bxy+cy^2$正定即可.所以$a,b,c$满足$b^2-4ac<0$ 且 $a>0$.
\end{proof}
\begin{exercise}[7]
求$x$轴绕直线$x=y=z-1$旋转所得旋转曲面的一般方程.  
\end{exercise}
\begin{proof}[\textbf{解}]
设$(x_1,0,0)$是$x$轴上的任意一个点.然后在零时刻,$x$轴开始绕直线
$x=y=z-1$运动.经过时间$t$后,$(x_1,0,0)$运动到了点
$(x_1(t),y_{1}(t),z_1(t))$.显然,
$$
(x_{1}-x_1(t),0-y_1(t),0-z_1(t))\cdot (1,1,1)=0,
$$
其中$(1,1,1)$是直线$x=y=z-1$的方向向量.也即,
\begin{equation}
  \label{eq:1}
  x_{1}=x_1(t)+y_1(t)+z_1(t).
\end{equation}
而且,$(x_1,0,0)$和$(x_1(t),y_1(t),z_1(t))$到直线$x=y=z-1$的距离相
等.下面我们来求点$q=(a,b,c)$到直线$x=y=z-1$的距离.点$p=(t,t,t+1)$位于直线
$x=y=z-1$上,设直线$pq$垂直于$x=y=z-1$,则有
$$
(a-t,b-t,c-t-1)\cdot (1,1,1)=0,
$$
也即,
$$
t=\frac{a+b+c-1}{3}.
$$
可见,
$$
  |\ov{pq}|^2=(\frac{b+c-2a-1}{3})^2+(\frac{a+c-2b-1}{3})^2+(\frac{a+b-2c+2}{3})^2.
$$
于是,我们有
$$
\frac{2x_1^2+2x_1+2}{3}=(\frac{y_1(t)+z_1(t)-2x_1(t)-1}{3})^2+(\frac{x_1(t)+z_1(t)-2y_1(t)-1}{3})^2+(\frac{x_1(t)+y_1(t)-2z_1(t)+2}{3})^2.
$$
将方程\eqref{eq:1}代入,化简可得
$$
x_{1}(t)y_{1}(t)+y_{1}(t)z_{1}(t)+z_{1}(t)x_{1}(t)+z_{1}(t)=0.
$$
于是,可得旋转曲面的方程为
$$
xy+yz+zx+z=0.
$$
\end{proof}
\end{document}