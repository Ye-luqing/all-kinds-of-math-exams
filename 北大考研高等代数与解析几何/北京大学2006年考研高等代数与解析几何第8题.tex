\documentclass[a4paper]{article}
\usepackage{amsmath,amsfonts,amsthm,amssymb}
\usepackage{bm}
\usepackage{hyperref}
\usepackage{geometry}
\usepackage{yhmath}
\usepackage{pstricks-add}
\usepackage{framed,mdframed}
\usepackage{graphicx,color} 
\usepackage{mathrsfs,xcolor} 
\usepackage[all]{xy}
\usepackage{fancybox} 
\usepackage{xeCJK}
\newtheorem*{theo}{定理}
\newtheorem*{exe}{题目}
\newtheorem*{rem}{评论}
\newtheorem*{lemma}{引理}
\newtheorem*{coro}{推论}
\newtheorem*{exa}{例}
\newenvironment{corollary}
{\bigskip\begin{mdframed}\begin{coro}}
    {\end{coro}\end{mdframed}\bigskip}
\newenvironment{theorem}
{\bigskip\begin{mdframed}\begin{theo}}
    {\end{theo}\end{mdframed}\bigskip}
\newenvironment{exercise}
{\bigskip\begin{mdframed}\begin{exe}}
    {\end{exe}\end{mdframed}\bigskip}
\newenvironment{example}
{\bigskip\begin{mdframed}\begin{exa}}
    {\end{exa}\end{mdframed}\bigskip}
\newenvironment{remark}
{\bigskip\begin{mdframed}\begin{rem}}
    {\end{rem}\end{mdframed}\bigskip}
\geometry{left=2.5cm,right=2.5cm,top=2.5cm,bottom=2.5cm}
\setCJKmainfont[BoldFont=SimHei]{SimSun}
\renewcommand{\today}{\number\year 年 \number\month 月 \number\day 日}
\newcommand{\D}{\displaystyle}\newcommand{\ri}{\Rightarrow}
\newcommand{\ds}{\displaystyle} \renewcommand{\ni}{\noindent}
\newcommand{\ov}{\overrightarrow}
\newcommand{\pa}{\partial} \newcommand{\Om}{\Omega}
\newcommand{\om}{\omega} \newcommand{\sik}{\sum_{i=1}^k}
\newcommand{\vov}{\Vert\omega\Vert} \newcommand{\Umy}{U_{\mu_i,y^i}}
\newcommand{\lamns}{\lambda_n^{^{\scriptstyle\sigma}}}
\newcommand{\chiomn}{\chi_{_{\Omega_n}}}
\newcommand{\ullim}{\underline{\lim}} \newcommand{\bsy}{\boldsymbol}
\newcommand{\mvb}{\mathversion{bold}} \newcommand{\la}{\lambda}
\newcommand{\La}{\Lambda} \newcommand{\va}{\varepsilon}
\newcommand{\be}{\beta} \newcommand{\al}{\alpha}
\newcommand{\dis}{\displaystyle} \newcommand{\R}{{\mathbb R}}
\newcommand{\N}{{\mathbb N}} \newcommand{\cF}{{\mathcal F}}
\newcommand{\gB}{{\mathfrak B}} \newcommand{\eps}{\epsilon}
\renewcommand\refname{参考文献}\renewcommand\figurename{图}
\usepackage[]{caption2} 
\renewcommand{\captionlabeldelim}{}
\setlength\parindent{0pt}
\begin{document}
\begin{exercise}
  证明四面体的每一个顶点到对面重心的线段共点,且这点到顶点的距离是它到
  对面重心距离的$3$倍.
\end{exercise}
\begin{proof}[\textbf{证明}]
首先,对于底面$BCD$上的任意一个点$P$来说,都有
\begin{equation}
  \label{eq:1}
  \ov{AP}=\lambda_1\ov{AB}+\lambda_2\ov{AC}+\lambda_3\ov{AD},
\end{equation}
其中
\begin{equation}
  \label{eq:2}
(\lambda_1,\lambda_2,\lambda_3)\in \{(x,y,z):x+y+z=1,0\leq x,y,z\leq 1\}.
\end{equation}
特别地,当$P$为$\triangle BCD$的重心$D_{BCD}$时,
\begin{equation}
  \label{eq:3}
  \ov{AD_{BCD}}=\frac{1}{3}\ov{AB}+\frac{1}{3}\ov{AC}+\frac{1}{3}\ov{AD}.
\end{equation}
对于线段$AD_{BCD}$上的任意一点$Q$来说,都有
\begin{equation}\label{eq:4}
\ov{AQ}=\lambda\ov{AD}=\frac{\lambda}{3}\ov{AB}+\frac{\lambda}{3}\ov{AC}+\frac{\lambda}{3}\ov{AD},
\end{equation}
其中$\lambda\in [0,1]$.同理,对于$\triangle ADB$的重心$D_{ADB}$来说,对
于线段$CD_{ADB}$上的任意一点$R$,都有
\begin{align}
  \ov{CR}&=\lambda\ov{CD_{ABD}}\\&=\frac{k}{3}\ov{CA}+\frac{k}{3}\ov{CB}+\frac{k}{3}\ov{CD}
\\&=-\frac{k}{3}\ov{AC}+\frac{k}{3}(\ov{AB}-\ov{AC})+\frac{k}{3}(\ov{AD}-\ov{AC})
\\&=\frac{k}{3}\ov{AB}-k\ov{AC}+\frac{k}{3}\ov{AD}.
\end{align}
由于
\begin{equation}\label{eq:5}
\ov{AR}=\ov{AC}+\ov{CR}=\frac{k}{3}\ov{AB}-(k-1)\ov{AC}+\frac{k}{3}\ov{AD},
\end{equation}
联立\eqref{eq:4},\eqref{eq:5},我们发现,当$\lambda=k=-(k-1)/3$,即
$\lambda=k=\frac{1}{4}$时,$\ov{AR}=\ov{AQ}$,此时$Q$和$R$重合.因此点$C$
与对面重心的连线和点$A$与对面重心的连线确实相交,交于$Q,R$.下面我们来证
明,直线$BQ$通过面$\triangle ACD$的重心即可证明四面体的每一个顶点到对面
重心的线段共点.这个证明是容易的,我们略去.且从式\eqref{eq:3}可以看出交
点到对面重心的距离是到顶点距离的三倍.
\end{proof}
\end{document}
