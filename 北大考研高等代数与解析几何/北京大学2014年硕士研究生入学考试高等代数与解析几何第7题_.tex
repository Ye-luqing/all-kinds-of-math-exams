\documentclass[a4paper]{article}
\usepackage{amsmath,amsfonts,amsthm,amssymb}
\usepackage{bm}
\usepackage{hyperref}
\usepackage{geometry}
\usepackage{yhmath}
\usepackage{pstricks-add}
\usepackage{framed,mdframed}
\usepackage{graphicx,color} 
\usepackage{mathrsfs,xcolor} 
\usepackage[all]{xy}
\usepackage{fancybox} 
\usepackage{xeCJK}
\newtheorem{theo}{定理}
\newtheorem*{exe}{题目}
\newtheorem*{rem}{评论}
\newmdtheoremenv{lemma}{引理}
\newmdtheoremenv{corollary}{推论}
\newmdtheoremenv{example}{例}
\newenvironment{theorem}
{\bigskip\begin{mdframed}\begin{theo}}
    {\end{theo}\end{mdframed}\bigskip}
\newenvironment{exercise}
{\bigskip\begin{mdframed}\begin{exe}}
    {\end{exe}\end{mdframed}\bigskip}
\geometry{left=2.5cm,right=2.5cm,top=2.5cm,bottom=2.5cm}
\setCJKmainfont[BoldFont=SimHei]{SimSun}
\renewcommand{\today}{\number\year 年 \number\month 月 \number\day 日}
\newcommand{\D}{\displaystyle}\newcommand{\ri}{\Rightarrow}
\newcommand{\ds}{\displaystyle} \renewcommand{\ni}{\noindent}
\newcommand{\pa}{\partial} \newcommand{\Om}{\Omega}
\newcommand{\om}{\omega} \newcommand{\sik}{\sum_{i=1}^k}
\newcommand{\vov}{\Vert\omega\Vert} \newcommand{\Umy}{U_{\mu_i,y^i}}
\newcommand{\lamns}{\lambda_n^{^{\scriptstyle\sigma}}}
\newcommand{\chiomn}{\chi_{_{\Omega_n}}}
\newcommand{\ullim}{\underline{\lim}} \newcommand{\bsy}{\boldsymbol}
\newcommand{\mvb}{\mathversion{bold}} \newcommand{\la}{\lambda}
\newcommand{\La}{\Lambda} \newcommand{\va}{\varepsilon}
\newcommand{\be}{\beta} \newcommand{\al}{\alpha}
\newcommand{\dis}{\displaystyle} \newcommand{\R}{{\mathbb R}}
\newcommand{\N}{{\mathbb N}} \newcommand{\cF}{{\mathcal F}}
\newcommand{\gB}{{\mathfrak B}} \newcommand{\eps}{\epsilon}
\renewcommand\refname{参考文献}\renewcommand\figurename{图}
\usepackage[]{caption2} 
\renewcommand{\captionlabeldelim}{}
\begin{document}
\title{\huge{\bf{北京大学2014年硕士研究生入学考试高等代数与解析几何第7
      题}}} \author{\small{叶卢
    庆\footnote{叶卢庆(1992---),男,杭州师范大学理学院数学与应用数学专业
      本科在读,E-mail:yeluqingmathematics@gmail.com}}}
\maketitle
\begin{exercise}[北京大学2014年硕士研究生入学考试高等代数与解析几何第7
  题]
求单叶双曲面
$$
\frac{x^2}{a^2}+\frac{y^2}{b^2}-\frac{z^2}{c^2}=1
$$
垂直的直母线交点的轨迹.  
\end{exercise}
\begin{proof}[\textbf{解}]
单叶双曲面可以分别化为
\begin{equation}\label{eq:1}
(\frac{x}{a})^2+(\frac{y}{b})^2=(\frac{z}{c})^2+1^2
\end{equation}
\begin{equation}
  \label{eq:2}
(\frac{x}{a})^2+(\frac{y}{b})^2=(\frac{z}{c})^2+(-1)^2.
\end{equation}
针对方程\eqref{eq:1},可以设
\begin{equation}
  \label{eq:3}
  \begin{pmatrix}
    \frac{z}{c}\\
1
  \end{pmatrix}=
  \begin{pmatrix}
    \cos\alpha&-\sin\alpha\\
\sin\alpha&\cos\alpha
  \end{pmatrix}
  \begin{pmatrix}
    \frac{x}{a}\\
\frac{y}{b}
  \end{pmatrix}.
\end{equation}
针对方程\eqref{eq:2},可以设
\begin{equation}
  \label{eq:4}
  \begin{pmatrix}
    \frac{z}{c}\\
-1
  \end{pmatrix}=
  \begin{pmatrix}
    \cos\beta&-\sin\beta\\
\sin\beta&\cos\beta
  \end{pmatrix}
  \begin{pmatrix}
    \frac{x}{a}\\
\frac{y}{b}
  \end{pmatrix}.
\end{equation}
无论是方程\eqref{eq:3},还是方程\eqref{eq:4},都是代表双曲面的直母线.其
中$\alpha,\beta\in \mathbf{R}$是参数.而
且易得形如\eqref{eq:3}和形如\eqref{eq:4}的直母线族都可以遍历双曲面上所
有点,而且易得直母线族\eqref{eq:3}里的任意两条直线异面,直母线族
\eqref{eq:4}里的任意两条直线也异面\footnote{见我的文档\href{http://blog.sciencenet.cn/home.php?mod=space&uid=604208&do=blog&id=833256}{《单叶双曲面和双曲抛物面是直纹面的另类证明》}}.因此只可能是直母线族\eqref{eq:3}
里的直线和直母线族\eqref{eq:4}里的直线相交.直
线\eqref{eq:3}的方向向量为
$$
(\frac{\sin\alpha}{a},\frac{\cos\alpha}{b},0)\times (\frac{\cos\alpha}{a},\frac{-\sin\alpha}{b},\frac{-1}{c})=(-\frac{\cos\alpha}{bc},\frac{\sin\alpha}{ac},\frac{-1}{ab}),
$$
直线\eqref{eq:4}的方向向量为
$$
(-\frac{\cos\beta}{bc},\frac{\sin\beta}{ac},\frac{-1}{ab}).
$$
当直线\eqref{eq:3}和\eqref{eq:4}垂直时,我们有
\begin{equation}
  \label{eq:5}
  \frac{\cos\alpha\cos\beta}{b^2c^2}+\frac{\sin\alpha\sin\beta}{a^2c^2}+\frac{1}{a^2b^2}=0.
\end{equation}
现设直线\eqref{eq:3}和\eqref{eq:4}的交点为$(x_0,y_0,z_0)$,则$$
\begin{cases}
\frac{x_0}{a}\cos\alpha-\frac{y_0}{b}\sin\alpha=\frac{z_0}{c},\\
\frac{y_0}{b}\cos\alpha+\frac{x_0}{a}\sin\alpha=1,\\
\frac{x_0}{a}\cos\beta-\frac{y_0}{b}\sin\beta=\frac{z_0}{c},\\
\frac{y_0}{b}\cos\beta+\frac{x_0}{a}\sin\beta=-1,\\
\end{cases}
$$
解得
$$
\cos\alpha=\frac{\frac{x_0z_0}{ac}+\frac{y_0}{b}}{\frac{x_0^2}{a^2}+\frac{y_0^2}{b^2}},\sin\alpha=\frac{\frac{x_0}{a}-\frac{y_0z_0}{bc}}{\frac{x_0^2}{a^2}+\frac{y_0^2}{b^2}}.
$$
$$
\cos\beta=\frac{\frac{x_0z_0}{ac}-\frac{y_0}{b}}{\frac{x_0^2}{a^2}+\frac{y_0^2}{b^2}},\sin\beta=\frac{-\frac{x_0}{a}-\frac{y_0z_0}{bc}}{\frac{x_0^2}{a^2}+\frac{y_0^2}{b^2}}.
$$
代入方程\eqref{eq:5},可得单叶双曲面互相垂直的直母线的交点轨迹为
$$
(\frac{x^2z^2}{c^2}-\frac{a^{2}y^2}{b^2})+(\frac{y^2z^2}{c^2}-\frac{b^{2}x^2}{a^2})+c^2(\frac{x^2}{a^2}+\frac{y^2}{b^2})^2=0.
$$
\end{proof}
\end{document}




