\documentclass[a4paper]{article}
\usepackage{amsmath,amsfonts,amsthm,amssymb}
\usepackage{bm}
\usepackage{draftwatermark,euler}
\SetWatermarkText{http://blog.sciencenet.cn/u/Yaleking}%设置水印文字
\SetWatermarkLightness{0.8}%设置水印亮度
\SetWatermarkScale{0.35}%设置水印大小
\usepackage{hyperref}
\usepackage{geometry}
\usepackage{yhmath}
\usepackage{pstricks-add}
\usepackage{framed,mdframed}
\usepackage{graphicx,color} 
\usepackage{mathrsfs,xcolor} 
\usepackage[all]{xy}
\usepackage{fancybox} 
\usepackage{xeCJK}
\newtheorem*{theo}{定理}
\newtheorem*{exe}{题目}
\newtheorem*{rem}{评论}
\newmdtheoremenv{lemma}{引理}
\newmdtheoremenv{corollary}{推论}
\newmdtheoremenv{example}{例}
\newenvironment{theorem}
{\bigskip\begin{mdframed}\begin{theo}}
    {\end{theo}\end{mdframed}\bigskip}
\newenvironment{exercise}
{\bigskip\begin{mdframed}\begin{exe}}
    {\end{exe}\end{mdframed}\bigskip}
\geometry{left=2.5cm,right=2.5cm,top=2.5cm,bottom=2.5cm}
\setCJKmainfont[BoldFont=SimHei]{SimSun}
\renewcommand{\today}{\number\year 年 \number\month 月 \number\day 日}
\newcommand{\D}{\displaystyle}\newcommand{\ri}{\Rightarrow}
\newcommand{\ds}{\displaystyle} \renewcommand{\ni}{\noindent}
\newcommand{\ov}{\overrightarrow}
\newcommand{\pa}{\partial} \newcommand{\Om}{\Omega}
\newcommand{\om}{\omega} \newcommand{\sik}{\sum_{i=1}^k}
\newcommand{\vov}{\Vert\omega\Vert} \newcommand{\Umy}{U_{\mu_i,y^i}}
\newcommand{\lamns}{\lambda_n^{^{\scriptstyle\sigma}}}
\newcommand{\chiomn}{\chi_{_{\Omega_n}}}
\newcommand{\ullim}{\underline{\lim}} \newcommand{\bsy}{\boldsymbol}
\newcommand{\mvb}{\mathversion{bold}} \newcommand{\la}{\lambda}
\newcommand{\La}{\Lambda} \newcommand{\va}{\varepsilon}
\newcommand{\be}{\beta} \newcommand{\al}{\alpha}
\newcommand{\dis}{\displaystyle} \newcommand{\R}{{\mathbb R}}
\newcommand{\N}{{\mathbb N}} \newcommand{\cF}{{\mathcal F}}
\newcommand{\gB}{{\mathfrak B}} \newcommand{\eps}{\epsilon}
\renewcommand\refname{参考文献}\renewcommand\figurename{图}
\usepackage[]{caption2} 
\renewcommand{\captionlabeldelim}{}
\setlength\parindent{0pt}
\begin{document}
\title{\huge{\bf{山师大2007年高等代数与解析几何第7题}}} \author{\small{叶卢庆\footnote{叶卢庆(1992---),男,杭州师范大学理学院数学与应用数学专业本科在读,E-mail:yeluqingmathematics@gmail.com}}}
\maketitle
\begin{exercise}
求单叶双曲面$x^2+y^2-z^2=1$过点$(1,3,3)$的两条直母线所决定的平面方程.
\end{exercise}
\begin{proof}[\textbf{解}]
$$
x^2+y^2-z^2=1\iff (x+z)(x-z)=(1+y)(1-y).
$$
因此
$$
\begin{cases}
  x+z=u(1+y),\\
u(x-z)=1-y.
\end{cases}
$$
或者
$$
\begin{cases}
  x+z=v(1-y),\\
v(x-z)=1+y.
\end{cases}
$$
其中$u,v$都是参数.这是单叶双曲面的两族直母线.如果是$u$族直母线通过点
$(1,3,3)$,则有
$$
\begin{cases}
  4=4u,\\
-2u=-2.
\end{cases}
$$
可得$u=1$.于是通过$(1,3,3)$的$u$族直母线为
$$
\begin{cases}
  x+z=1+y,\\
x-z=1-y.
\end{cases}\iff
\begin{cases}
  x=1,\\
z=y
\end{cases}.
$$
可见该直线的方向向量为$(0,1,1)$.如果是$v$族直母线通过点$(1,3,3)$,则
$$
\begin{cases}
  4=v(-2),\\
v(-2)=4.
\end{cases}
$$
可得$v=-2$,于是通过$(1,3,3)$的$v$族直母线为
$$
\begin{cases}
  x+z=-2(1-y),\\
-2(x-z)=1+y
\end{cases}\iff \frac{x}{1}=\frac{y-\frac{5}{3}}{\frac{4}{3}}=\frac{z-\frac{4}{3}}{\frac{5}{3}}.
$$
可见该直线的方向向量为$(3,4,5)$.于是两条直线展成的平面的法向量为
$$
(0,1,1)\times (3,4,5)=(1,3,-3).
$$
于是平面方程为
$$
x+3y-3z-1=0.
$$
\end{proof}
\end{document}
